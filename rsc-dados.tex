% define valor para variaveis
\nomeservidor{Servidor fulano de tal}
\datanascimento{aa/mm/aaaa}
\siape{1111111}
\email{email@ifsudestemg.edu.br}
\nivel{01}
\classe{D1}
\dataingressoservicopublico{dd/mm/aaaa}
\dataingressoif{dd/mm/aaaa}
\formacao{Graduação em Latex\\Mestrado em desenvolvimento de modelos em Latex}
\tempoefetivoexercicio{N meses}
\rscpretendida{3}
\cpf{111.111.111-11}
\instituicao{Instituto Federal de Educação, Ciência e Tecnologia do Sudeste de Minas Gerais}
\campus{Santos Dumont}
\genero{masculino} % masculino ou feminino

% Descomente essa linha quando terminar de preencher o formulário para remover os comentários nas capas das páginas de comprovantes.
%\removercomentarios{}

% preenche os valores das pontuacoes
% RSC-I
\rscipontosi{0.25}
\rscipontosii{1}
\rscipontosiii{2}
\rscipontosiv{3}
\rscipontosv{4}
\rscipontosvi{5}
\rscipontosvii{6}
\rscipontosviii{7}
\pgfmathsetmacro{\rscipontostotal}{
	\the\rscipontosi +
	\the\rscipontosii +
	\the\rscipontosiii +
	\the\rscipontosiv +
	\the\rscipontosv +
	\the\rscipontosvi +
	\the\rscipontosvii +
	\the\rscipontosviii
}

% RSC-II
\rsciipontosi{0}
\rsciipontosii{1}
\rsciipontosiii{2}
\rsciipontosiv{3}
\rsciipontosv{4}
\rsciipontosvi{5}
\rsciipontosvii{6}
\pgfmathsetmacro{\rsciipontostotal}{
	\the\rsciipontosi +
	\the\rsciipontosii +
	\the\rsciipontosiii +
	\the\rsciipontosiv +
	\the\rsciipontosv +
	\the\rsciipontosvi +
	\the\rsciipontosvii
}

% RSC-III
\rsciiipontosi{0}
\rsciiipontosii{1}
\rsciiipontosiii{2}
\rsciiipontosiv{3}
\rsciiipontosv{4}
\rsciiipontosvi{5}
\rsciiipontosvii{6}
\pgfmathsetmacro{\rsciiipontostotal}{
	\the\rsciiipontosi +
	\the\rsciiipontosii +
	\the\rsciiipontosiii +
	\the\rsciiipontosiv +
	\the\rsciiipontosv +
	\the\rsciiipontosvi +
	\the\rsciiipontosvii
}

\pgfmathsetmacro{\rscpontostotal}{
	\rscipontostotal +
	\rsciipontostotal +
	\rsciiipontostotal
}

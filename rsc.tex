\documentclass[12pt,a4paper]{article}
\usepackage[a4paper,bindingoffset=0in,%
            left=0.5in,right=0.5in,top=1in,bottom=2in,%
            footskip=.25in]{geometry}
\usepackage[brazil,english]{babel}
%\usepackage[utf8]{inputenc}
\usepackage[utf8x]{inputenc}
\usepackage[T1]{fontenc}
\usepackage{lipsum}%% a garbage package you don't need except to create examples.
\usepackage{fancyhdr}
\usepackage{graphicx}
\usepackage{xcolor}
\usepackage{makecell}
\usepackage{helvet}
\renewcommand{\familydefault}{\sfdefault}
\usepackage{blindtext}
\usepackage{amsmath}
\usepackage{changepage}
\usepackage{tikz} % pra fazer conta com variavel
\usepackage{xintexpr} % pra fazer conta com decimal
\usepackage{caption}
\usepackage{tocloft}
\usepackage{etoolbox}
\usepackage{pdfpages} % para incluir pdfs
\usepackage{refcount}

\cftsetpnumwidth{1.0cm}

\renewcommand{\cftsecdotsep}{\cftdotsep}
\renewcommand{\cftsecleader}{\normalfont\cftdotfill{\cftsecdotsep}}


\renewcommand{\baselinestretch}{1.5}

\addto\captionsbrazil{%
	\renewcommand{\contentsname}%
		{Sumário Relatório Descritivo}
}

\addto\captionsenglish{%
	\renewcommand{\contentsname}%
		{Sumário Relatório Descritivo}
}

\pagestyle{fancy}

\makeatletter
\newcommand\Rodape[1]{\def\@Rodape{#1}}

\fancyhead[LE,LO]{}
\fancyhead[CE,CO]{}
\fancyhead[RE,RO]{}
\fancyhead[LE,LO]{\includegraphics[width=5cm]{logo_ifsudeste.jpg}\hfill\includegraphics[width=2cm]{brasao-brasil.png}\\{\large \textbf{FORMULÁRIOS RECONHECIMENTO DE SABERES E COMPETÊNCIAS – RSC}}}
\setlength\headheight{80pt}
%\renewcommand{\headrulewidth}{0pt}
%\renewcommand{\headrulewidth}{0.4pt}
%\renewcommand{\footrulewidth}{0.4pt}

\fancyfoot{}
% numeracao a direita
\Rodape{\thepage}
\fancyfoot[R]{\@Rodape}

% para o fancy funcionar na pagina com tableofcontents
\fancypagestyle{plain}{%
	\fancyhf{}
	\fancyhead[LE,LO]{}
	\fancyhead[CE,CO]{}
	\fancyhead[RE,RO]{}
	\fancyhead[LE,LO]{\includegraphics[width=5cm]{logo_ifsudeste.jpg}\hfill\includegraphics[width=2cm]{brasao-brasil.png}\\{\large 					\textbf{FORMULÁRIOS RECONHECIMENTO DE SABERES E COMPETÊNCIAS – RSC}}}
	\setlength\headheight{80pt}
	%\renewcommand{\headrulewidth}{0pt}
	%\renewcommand{\headrulewidth}{0.4pt}
	%\renewcommand{\footrulewidth}{0.4pt}
	\fancyfoot{}
	\fancyfoot[R]{\@Rodape}
}

% para o fancy funcionar na pagina com tableofcontents
\fancypagestyle{plainpdf}{%
	\fancyhf{}
	\renewcommand{\headrulewidth}{0pt}
	\fancyfoot{}
	\fancyfoot[R]{\@Rodape}
}

\newcommand{\nada}{
	{\color{white}NADA}
}

\begin{document}

\pagenumbering{roman} % bota a numeracao de pagina em romano

\renewcommand{\cftsecdotsep}{\cftdotsep}
\renewcommand{\cftsecleader}{\normalfont\cftdotfill{\cftsecdotsep}}

\makeatletter
\newcommand\Rodape[1]{\def\@Rodape{#1}}

\fancyhead[LE,LO]{}
\fancyhead[CE,CO]{}
\fancyhead[RE,RO]{}
\fancyhead[LE,LO]{\includegraphics[width=5cm]{nao-editar/logo_ifsudeste.jpg}\hfill\includegraphics[width=2cm]{nao-editar/brasao-brasil.png}\\{\large \textbf{FORMULÁRIOS RECONHECIMENTO DE SABERES E COMPETÊNCIAS – RSC}}}
\setlength\headheight{80pt}
%\renewcommand{\headrulewidth}{0pt}
%\renewcommand{\headrulewidth}{0.4pt}
%\renewcommand{\footrulewidth}{0.4pt}

\fancyfoot{}
% numeracao a direita
\Rodape{\thepage}
\fancyfoot[R]{\@Rodape}

% para o fancy funcionar na pagina com tableofcontents
\fancypagestyle{plain}{%
	\fancyhf{}
	\fancyhead[LE,LO]{}
	\fancyhead[CE,CO]{}
	\fancyhead[RE,RO]{}
	\fancyhead[LE,LO]{\includegraphics[width=5cm]{nao-editar/logo_ifsudeste.jpg}\hfill\includegraphics[width=2cm]{nao-editar/brasao-brasil.png}\\{\large 					\textbf{FORMULÁRIOS RECONHECIMENTO DE SABERES E COMPETÊNCIAS – RSC}}}
	\setlength\headheight{80pt}
	%\renewcommand{\headrulewidth}{0pt}
	%\renewcommand{\headrulewidth}{0.4pt}
	%\renewcommand{\footrulewidth}{0.4pt}
	\fancyfoot{}
	\fancyfoot[R]{\@Rodape}
}

\newcommand\IndicacaoPontuacaoRSC[1]{\def\@IndicacaoPontuacaoRSC{#1}}

% para o fancy funcionar na pagina com tableofcontents
\fancypagestyle{plainpdf}{%
	\fancyhf{}
	\fancyhead[LE,LO]{}
	\fancyhead[CE,CO]{}
	\fancyhead[RE,RO]{}
	\fancyhead[RE,RO]{\textbf{{\Large \@IndicacaoPontuacaoRSC}}}
	\renewcommand{\headrulewidth}{0pt}
	\fancyfoot{}
	\fancyfoot[R]{\@Rodape}
	%\setlength\headheight{0pt}
	\newgeometry{bindingoffset=0in,left=0.5in,right=0.5in,top=1in,bottom=1in,footskip=1.45in}
	%fancyhfoffset[E,O]{0pt}
}

\newcommand{\nada}{
	{\color{white}NADA}
}

\newcommand{\paginainicialpdf}[1]{
	\pageref{inicio-#1}
}

\newcommand{\paginafinalpdf}[1]{
	\subtrair{\pageref{fim-#1}}{1}
}

\newcommand{\incluirpdf}[3]{
	\IndicacaoPontuacaoRSC{#2}
	\label{inicio-#3}
	\includepdf[pages=#1, offset=0 -50, pagecommand={\thispagestyle{plainpdf}}]{#3}
	\label{fim-#3}
}

% define variaveis
\newtoks\nomeservidor
\newtoks\datanascimento
\newtoks\siape
\newtoks\email
\newtoks\nivel
\newtoks\classe
\newtoks\dataingressoservicopublico
\newtoks\dataingressoif
\newtoks\formacao
\newtoks\tempoefetivoexercicio
\newtoks\rscpretendida
\newtoks\cpf
\newtoks\campus
\newtoks\instituicao

% define as variaveis pontuacoes
% RSC-I
\newtoks\rscipontosi
\newtoks\rscipontosii
\newtoks\rscipontosiii
\newtoks\rscipontosiv
\newtoks\rscipontosv
\newtoks\rscipontosvi
\newtoks\rscipontosvii
\newtoks\rscipontosviii
%\newtoks\rscipontostotal

% RSC-II
\newtoks\rsciipontosi
\newtoks\rsciipontosii
\newtoks\rsciipontosiii
\newtoks\rsciipontosiv
\newtoks\rsciipontosv
\newtoks\rsciipontosvi
\newtoks\rsciipontosvii
\newtoks\rsciipontostotal

% RSC-II
\newtoks\rsciiipontosi
\newtoks\rsciiipontosii
\newtoks\rsciiipontosiii
\newtoks\rsciiipontosiv
\newtoks\rsciiipontosv
\newtoks\rsciiipontosvi
\newtoks\rsciiipontosvii
\newtoks\rsciiipontostotal

% FUNCOES
\newcommand{\diferencapagina}[2]{%
\number\numexpr\getpagerefnumber{#1}-#2\relax
}


% define valor para variaveis
\nomeservidor{Servidor fulano de tal}
\datanascimento{aa/mm/aaaa}
\siape{1111111}
\email{email@ifsudestemg.edu.br}
\nivel{01}
\classe{D1}
\dataingressoservicopublico{dd/mm/aaaa}
\dataingressoif{dd/mm/aaaa}
\formacao{Graduação em Latex\\Mestrado em desenvolvimento de modelos em Latex}
\tempoefetivoexercicio{N meses}
\rscpretendida{3}
\cpf{111.111.111-11}
\instituicao{Instituto Federal de Educação, Ciência e Tecnologia do Sudeste de Minas Gerais}
\campus{Santos Dumont}
\genero{masculino} % masculino ou feminino

% Descomente essa linha quando terminar de preencher o formulário para remover os comentários nas capas das páginas de comprovantes.
%\removercomentarios{}

% preenche os valores das pontuacoes
% RSC-I
\rscipontosi{0.25}
\rscipontosii{1}
\rscipontosiii{2}
\rscipontosiv{3}
\rscipontosv{4}
\rscipontosvi{5}
\rscipontosvii{6}
\rscipontosviii{7}
\pgfmathsetmacro{\rscipontostotal}{
	\the\rscipontosi +
	\the\rscipontosii +
	\the\rscipontosiii +
	\the\rscipontosiv +
	\the\rscipontosv +
	\the\rscipontosvi +
	\the\rscipontosvii +
	\the\rscipontosviii
}

% RSC-II
\rsciipontosi{0}
\rsciipontosii{1}
\rsciipontosiii{2}
\rsciipontosiv{3}
\rsciipontosv{4}
\rsciipontosvi{5}
\rsciipontosvii{6}
\pgfmathsetmacro{\rsciipontostotal}{
	\the\rsciipontosi +
	\the\rsciipontosii +
	\the\rsciipontosiii +
	\the\rsciipontosiv +
	\the\rsciipontosv +
	\the\rsciipontosvi +
	\the\rsciipontosvii
}

% RSC-III
\rsciiipontosi{0}
\rsciiipontosii{1}
\rsciiipontosiii{2}
\rsciiipontosiv{3}
\rsciiipontosv{4}
\rsciiipontosvi{5}
\rsciiipontosvii{6}
\pgfmathsetmacro{\rsciiipontostotal}{
	\the\rsciiipontosi +
	\the\rsciiipontosii +
	\the\rsciiipontosiii +
	\the\rsciiipontosiv +
	\the\rsciiipontosv +
	\the\rsciiipontosvi +
	\the\rsciiipontosvii
}

\pgfmathsetmacro{\rscpontostotal}{
	\rscipontostotal +
	\rsciipontostotal +
	\rsciiipontostotal
}
% Não esqueça de preencher esse arquivo

\begin{center}
\textbf{{\large ANEXO I - FORMULÁRIO PARA SOLICITAÇÃO DE RSC}}
\end{center}
%
% Tabela
%

\begin{table}[ht]
	\centering
	%\begin{adjustwidth}{-0.18in}{}
	\begin{tabular}{|l|}
		\hline
		\makebox[1\textwidth]{\makecell[l]{Nome do servidor\expandafter\ifstrequal\expandafter{\the\genero}{feminino}{(a)}{}:\\\the\nomeservidor} \hfill \makecell[l]{Data de Nascimento:\\\the\datanascimento}}\\
		\hline
		\makebox[1\textwidth]{\makecell[l]{E-mail institucional:\\\the\email}}\\
		\hline
		\makebox[1\textwidth]{
			\makecell[l]{SIAPE:\\\the\siape} \hfill \makecell[l]{Classe:\\\the\classe}
			\hfill \makecell[l]{Nível:\\\the\nivel}
		}\\
		\hline
		\makebox[1\textwidth]{
			\makecell[l]{Data de ingresso no Serviço Público Federal:\\\the\dataingressoservicopublico \space(posse)} \hfill 
			\makecell[l]{Data de Ingresso no IF Sudeste MG:\\\the\dataingressoif \space(exercício)}
		}\\
		\hline
		\makebox[1\textwidth]{\makecell[l]{Formação (graduação e pós-graduação, se houver):\\\the\formacao}}\\
		\hline
		\makebox[1\textwidth]{\makecell[l]{Tempo efetivo de exercício (descontado ausências e licenciamentos não previstos na\\ legislação vigente):\\\the\tempoefetivoexercicio}}\\
		\hline
		\makebox[1\textwidth]{\makecell[l]{RSC pretendida:}
			\makecell[l]{(\ifnum\the\rscpretendida=1 \underline{X}\else \underline{\hspace{0.1in}}\fi) RSC I} \hfill
			\makecell[l]{(\ifnum\the\rscpretendida=2 \underline{X}\else \underline{\hspace{0.1in}}\fi) RSC II} \hfill
			\makecell[l]{(\ifnum\the\rscpretendida=3 \underline{X}\else \underline{\hspace{0.1in}}\fi) RSC III}
		}\\
		\hline
	\end{tabular}
	%\end{adjustwidth}
\end{table}


Eu \textbf{\underline{\the\nomeservidor}}, professor\expandafter\ifstrequal\expandafter{\the\genero}{feminino}{a}{} da carreira de Ensino Básico, Técnico e Tecnológico (EBTT), SIAPE \textbf{\underline{\the\siape}}, CPF \textbf{\underline{\the\cpf}}, venho solicitar à Subcomissão Permanente de Pessoal Docente do campus \textbf{\underline{\the\campus}}, do IF Sudeste MG o recebimento e o encaminhamento do meu relatório para fins de Concessão de RSC conforme na Lei nº 12.772, de 28/12/2012, na Lei nº 12.863, de 24/09/2013.
\\
\\
Data: \underline{\hspace{0.3in}} / \underline{\hspace{0.3in}} / \underline{\hspace{0.3in}}
\\
\\
\\
\underline{\hspace{5in}}\\
Assinatura do Servidor Requerente




\begin{center}
\textbf{{\large ANEXO II - FORMULÁRIO INDICATIVO DE PONTUAÇÃO RSCs}}
\end{center}
%
% Tabela
%
\begin{table}[ht]
	\centering
	%\begin{adjustwidth}{-0.18in}{}
	\begin{tabular}{|l|}
		\hline
		\makebox[1\textwidth]{\makecell[l]{Nome do servidor:\\\the\nomeservidor} \hfill \makecell[l]{Data de Nascimento:\\\the\datanascimento}}\\
		\hline
		\makebox[1\textwidth]{
			\makecell[l]{SIAPE:\\\the\siape} \hfill \makecell[l]{Classe:\\\the\classe}
			\hfill \makecell[l]{Nível:\\\the\nivel}
		}\\
		\hline
		\makebox[1\textwidth]{\makecell[l]{RSC pretendida:}
			\makecell[l]{(\ifnum\the\rscpretendida=1 \underline{X}\else \underline{\hspace{0.1in}}\fi) RSC I} \hfill
			\makecell[l]{(\ifnum\the\rscpretendida=2 \underline{X}\else \underline{\hspace{0.1in}}\fi) RSC II} \hfill
			\makecell[l]{(\ifnum\the\rscpretendida=3 \underline{X}\else \underline{\hspace{0.1in}}\fi) RSC III}
		}\\
		\hline
	\end{tabular}
	%\end{adjustwidth}	
\end{table}

Eu \textbf{\underline{\the\nomeservidor}}, professor\expandafter\ifstrequal\expandafter{\the\genero}{feminino}{a}{} da carreira de Ensino Básico, Técnico e Tecnológico (EBTT), SIAPE \textbf{\underline{\the\siape}}, CPF \textbf{\underline{\the\cpf}}, após realização de retrospecto das minhas atividades profissionais e do arrolamento de dados preenchidos apresento à Comissão Especial de Avaliação de RSC o quadro abaixo sintetizando a pontuação obtida com minhas atividades.
\\
\begin{table}[ht]
	\centering
	\caption*{RSC I - Descrição sucinta das atividades desempenhadas}
	\begin{tabular}{|c|c|c|c|}
		\hline
		Diretriz & \makecell{Pontuação Obtida} & Peso & \makecell{Pontuação Máxima}\\
		\hline
		I & \xintifboolexpr{\the\rscipontosi < 10} {\the\rscipontosi} {10} & 1 & 10\\
		\hline
		II & \xintifboolexpr{\the\rscipontosii < 10} {\the\rscipontosii} {10} & 1 & 10\\
		\hline
		III & \xintifboolexpr{\the\rscipontosiii < 30} {\the\rscipontosiii} {30} & 3 & 30\\
		\hline
		IV & \xintifboolexpr{\the\rscipontosiv < 10} {\the\rscipontosiv} {10} & 1 & 10\\
		\hline
		V & \xintifboolexpr{\the\rscipontosv < 10} {\the\rscipontosv} {10} & 1 & 10\\
		\hline
		VI & \xintifboolexpr{\the\rscipontosvi < 10} {\the\rscipontosvi} {10} & 1 & 10\\
		\hline
		VII & \xintifboolexpr{\the\rscipontosvii < 10} {\the\rscipontosvii} {10} & 1 & 10\\
		\hline
		VIII & \xintifboolexpr{\the\rscipontosviii < 10} {\the\rscipontosviii} {10} & 1 & 10\\
		\hline
		\textbf{Total} & \xintifboolexpr{\rscipontostotal < 100} {\textbf{\rscipontostotal}} {\textbf{100}} & \textbf{-} & \textbf{100}\\
		\hline
	\end{tabular}
\end{table}

\newpage

\begin{table}[ht!p]
	\centering
	\caption*{RSC II - Descrição sucinta das atividades desempenhadas}
	\begin{tabular}{|c|c|c|c|}
		\hline
		Diretriz & \makecell{Pontuação Obtida} & Peso & \makecell{Pontuação Máxima}\\
		\hline
		I & \xintifboolexpr{\the\rsciipontosi < 20} {\the\rsciipontosi} {20} & 2 & 20\\
		\hline
		II & \xintifboolexpr{\the\rsciipontosii < 10} {\the\rsciipontosii} {10} & 1 & 10\\
		\hline
		III & \xintifboolexpr{\the\rsciipontosiii < 10} {\the\rsciipontosiii} {10} & 1 & 10\\
		\hline
		IV & \xintifboolexpr{\the\rsciipontosiv < 20} {\the\rsciipontosiv} {20} & 2 & 20\\
		\hline
		V & \xintifboolexpr{\the\rsciipontosv < 10} {\the\rsciipontosv} {10} & 1 & 10\\
		\hline
		VI & \xintifboolexpr{\the\rsciipontosvi < 20} {\the\rsciipontosvi} {20} & 2 & 20\\
		\hline
		VII & \xintifboolexpr{\the\rsciipontosvii < 10} {\the\rsciipontosvii} {10} & 1 & 10\\
		\hline
		\textbf{Total} & \xintifboolexpr{\rsciipontostotal < 100} {\textbf{\rsciipontostotal}} {\textbf{100}} & \textbf{-} & \textbf{100}\\
		\hline
	\end{tabular}
\end{table}

\begin{table}[ht!p]
	\centering
	\caption*{RSC III - Descrição sucinta das atividades desempenhadas}
	\begin{tabular}{|c|c|c|c|}
		\hline
		Diretriz & \makecell{Pontuação Obtida} & Peso & \makecell{Pontuação Máxima}\\
		\hline
		I & \xintifboolexpr{\the\rsciiipontosi < 10} {\the\rsciiipontosi} {10} & 1 & 10\\
		\hline
		II & \xintifboolexpr{\the\rsciiipontosii < 20} {\the\rsciiipontosii} {20} & 2 & 20\\
		\hline
		III & \xintifboolexpr{\the\rsciiipontosiii < 20} {\the\rsciiipontosiii} {20} & 2 & 20\\
		\hline
		IV & \xintifboolexpr{\the\rsciiipontosiv < 10} {\the\rsciiipontosiv} {10} & 1 & 10\\
		\hline
		V & \xintifboolexpr{\the\rsciiipontosv < 10} {\the\rsciiipontosv} {10} & 1 & 10\\
		\hline
		VI & \xintifboolexpr{\the\rsciiipontosvi < 10} {\the\rsciiipontosvi} {10} & 1 & 10\\
		\hline
		VII & \xintifboolexpr{\the\rsciiipontosvii < 20} {\the\rsciiipontosvii} {20} & 2 & 20\\
		\hline
		\textbf{Total} & \xintifboolexpr{\rsciiipontostotal < 100} {\textbf{\rsciiipontostotal}} {\textbf{100}} & \textbf{-} & \textbf{100}\\
		\hline
	\end{tabular}
\end{table}

Em síntese, a pontuação obtida no nível RSC pretendido foi \ifnum\the\rscpretendida=1 \rscipontostotal\else \ifnum\the\rscpretendida=2 \rsciipontostotal\else \ifnum\the\rscpretendida=3 \rsciiipontostotal \fi \fi \fi \space pontos e a pontuação total obtida foi \rscpontostotal \space pontos.
\\
Data: \underline{\hspace{0.3in}} / \underline{\hspace{0.3in}} / \underline{\hspace{0.3in}}
\\
\underline{\hspace{5in}}\\
Assinatura do Servidor Requerente

\begin{center}
\textbf{\textit{{\large SUMÁRIO DO PROCESSO RSC}}}
\end{center}
%
% Tabela
%
% MUDAR 0 0(ZERO) NAS FUNCOES \DIFERENCAPAGINA PARA 1(UM) QUANDO ADICIONAR ALGUM COMPROVANTE
\begin{table}[ht]
	\centering
	\begin{tabular}{|c|c|c|}
		\hline
		\textbf{Descrição} & \textbf{Página Inicial} & \textbf{Página Final}\\
		\hline
		Planilha Simulação Pontuação & iv.1 & iv.5\\
		\hline
		Relatório Descritivo & \pageref{relatorio-descritivo}.1 & \pageref{relatorio-descritivo}.\diferencapagina{antes-documentacao-probatoria-formacao}{1}\\
		\hline
		\makecell{Documentação Comprobatória\\Formação Acadêmica e Complementar} & \pageref{documentacao-probatoria-formacao}.1 & \pageref{documentacao-probatoria-formacao}.\diferencapagina{antes-documentacao-probatoria-rsc-i}{0}\\
		\hline
		Documentação Comprobatória RSC I & \pageref{documentacao-probatoria-rsc-i}.1 & \pageref{documentacao-probatoria-rsc-i}.\diferencapagina{antes-documentacao-probatoria-rsc-ii}{0}\\
		\hline
		Documentação Comprobatória RSC II & \pageref{documentacao-probatoria-rsc-ii}.1 & \pageref{documentacao-probatoria-rsc-ii}.\diferencapagina{antes-documentacao-probatoria-rsc-iii}{0}\\
		\hline
		Documentação Comprobatória RSC III & \pageref{documentacao-probatoria-rsc-iii}.1 & \pageref{documentacao-probatoria-rsc-iii}.\diferencapagina{fim-documento}{0}\\
		\hline
	\end{tabular}
\end{table}

\ifcomentarios
{\color{red}
ORIENTAÇÕES DE PREENCHIMENTO:
\begin{enumerate}
	\itemsep-1em
	\item A montagem e a organização deste processo são de exclusiva responsabilidade do docente requerente;
	\item Depois de feita a simulação de pontuação, deve-se imprimir a planilha e anexá-la nesta documentação;
	\item É necessário seguir a nomenclatura de numeração contida no sumário acima, assim cada documento deve ser numerado com um algarismo romano indicando a parte a que pertence, seguido da numeração ordinal de sua sequência; 
	\item A numeração ordinal de páginas deve ser feita na parte inferior direita;
	\item Antes das documentações comprobatórias de cada RSC deverá existir uma capa indicativa do RSC I, II e III como já consta neste documento.
	\item Cada documento comprobatório RSC deverá ter a indicação da característica de pontuação a que se se refere, por exemplo, no canto superior direito da folha escreva: RSC I – 11, assim o avaliador saberá facilmente que tal documento se refere à atuação como conferencista ou palestrante;
	\item Após anexados os documentos, não se esqueça de preencher a numeração de página final referente a cada parte deste processo, no sumário do processo RSC (tabela acima – CAMPO PÁGINA FINAL).
\end{enumerate}
}
\fi

\label{antes-relatorio-descritivo}
\label{relatorio-descritivo}

\begin{center}
\textbf{{\Large Relatório Descritivo}}
\end{center}
%
{\Large
Nome: \the\nomeservidor
\\
\vspace{0.6in}
\\
CPF: \the\cpf
\\
\vspace{0.6in}
\\
SIAPE: \the\siape
\\
\vspace{0.6in}
\\
IDENTIFICAÇÃO INSTITUIÇÃO / CAMPUS: \the\instituicao \space Campus \the\campus
\\
\vspace{0.6in}
\\
Data: \underline{\hspace{0.3in}} / \underline{\hspace{0.3in}} / \underline{\hspace{0.3in}}
}

% Inicio conteudo relatorio descritivo
\pagenumbering{arabic}
\setcounter{page}{1}
\Rodape{Relatório Descritivo \hfill \pageref{relatorio-descritivo}.\thepage}

\begin{center}
\textbf{{\Large Relatório Descritivo}}
\end{center}
%

\tableofcontents

\section{Descrição do itinerário de formação, aperfeiçoamento e titulação;}
%\section{Descri\c{c}\~{a}o do itiner\'{a}rio de forma\c{c}\~{a}o, aperfei\c{c}oamento e titula\c{c}\~{a}o;}

Nenhum item a ser considerado.

\section{Descrição da atuação docente (ensino, pesquisa, inovação e extensão);}
%\section{Descri\c{c}\~{a}o da atua\c{c}\~{a}o docente (ensino, pesquisa, inova\c{c}\~{a}o e extens\~{a}o);}

Nenhum item a ser considerado.

\section{Indicação e descrição de produção acadêmica, técnico-científica, literária e/ou artística;}

Nenhum item a ser considerado.

\section{Descrição de atividades de prestação de serviços à comunidade;}

Nenhum item a ser considerado.

\section{Indicação e descrição de atividades de administração;}

Nenhum item a ser considerado.

\section{Indicação de títulos, prêmios e/ou aprovações em concursos;}

Nenhum item a ser considerado.

\label{antes-documentacao-probatoria-formacao}

\nada

\Rodape{\thepage}
\pagenumbering{roman}
\setcounter{page}{6}
% Fim conteudo relatorio descritivo


\label{documentacao-probatoria-formacao}

\begin{center}
	\begin{LARGE}
		\textbf{Documentação Comprobatória\\
		\vspace{0.2in}
		Formação Acadêmica e Complementar}
	\end{LARGE}
\end{center}

\ifcomentarios
\begin{center}
	{\color{red}(somente diploma de graduação e o diploma que gerou sua atual RT)}
\end{center}
\fi

% Inicio conteudo documentacao probatoria de formacao
\pagenumbering{arabic}
\setcounter{page}{1}
\Rodape{\pageref{documentacao-probatoria-formacao}.\thepage}

% \include com documentos probatorios da formacao

%\includepdf[pages=-, pagecommand={\thispagestyle{plainpdf}}]{certificado.pdf}

\label{antes-documentacao-probatoria-rsc-i}

\nada

\Rodape{\thepage}
\pagenumbering{roman}
\setcounter{page}{7}
% Fim conteudo documentacao probatoria de formacao


\label{documentacao-probatoria-rsc-i}

\begin{center}
	\begin{LARGE}
		\textbf{Documentação Comprobatória RSC I}
	\end{LARGE}
\end{center}

\begin{center}
	{\color{red}
		( Cada documento comprobatório RSC deverá ter a indicação da característica de pontuação a que se se refere, por exemplo, no canto superior direito da folha escreva: RSC I – 11, assim o avaliador saberá que tal documento se refere à atuação como conferencista ou palestrante )
	}
\end{center}

% Inicio conteudo documentacao probatoria RSC-I
\pagenumbering{arabic}
\setcounter{page}{1}
\Rodape{\pageref{documentacao-probatoria-rsc-i}.\thepage}

% \include com documentacao probatorio do RSC-I

%\includepdf[pages=-, pagecommand={\thispagestyle{plainpdf}}]{certificado.pdf}

\label{antes-documentacao-probatoria-rsc-ii}

\nada

\Rodape{\thepage}
\pagenumbering{roman}
\setcounter{page}{8}
% Fim conteudo documentacao probatoria RSC-I


\label{documentacao-probatoria-rsc-ii}

\begin{center}
	\begin{LARGE}
		\textbf{Documentação Comprobatória RSC II}
	\end{LARGE}
\end{center}

\ifcomentarios
\begin{center}
	{\color{red}
		( Cada documento comprobatório RSC deverá ter a indicação da característica de pontuação a que se se refere, por exemplo, no canto superior direito da folha escreva: RSC II – 58, assim o avaliador saberá que tal documento se refere à orientação ou coorientação  de TCC de cursos de graduação. )
	}
\end{center}
\fi

% Inicio conteudo documentacao probatoria RSC-II
\pagenumbering{arabic}
\setcounter{page}{1}
\Rodape{\pageref{documentacao-probatoria-rsc-ii}.\thepage}

% \include com documentacao probatorio do RSC-II

%\includepdf[pages=-, pagecommand={\thispagestyle{plainpdf}}]{certificado.pdf}

\label{antes-documentacao-probatoria-rsc-iii}

\nada

\Rodape{\thepage}
\pagenumbering{roman}
\setcounter{page}{9}
% Fim conteudo documentacao probatoria RSC-II


\label{documentacao-probatoria-rsc-iii}

\begin{center}
	\begin{LARGE}
		\textbf{Documentação Comprobatória RSC III}
	\end{LARGE}
\end{center}

\begin{center}
	{\color{red}
		( Cada documento comprobatório RSC deverá ter a indicação da característica de pontuação a que se se refere, por exemplo, no canto superior direito da folha escreva: RSC III – 102, assim o avaliador saberá que tal documento se refere à coordenação de núcleo de inovação tecnológica. )
	}
\end{center}

% Inicio conteudo documentacao probatoria RSC-III
\pagenumbering{arabic}
\setcounter{page}{1}
\Rodape{\pageref{documentacao-probatoria-rsc-iii}.\thepage}

% \include com documentacao probatorio do RSC-III

%\includepdf[pages=-, pagecommand={\thispagestyle{plainpdf}}]{certificado.pdf}

\include{empty/fim-documento}

\Rodape{\thepage}
\pagenumbering{roman}
\setcounter{page}{10}
% Fim conteudo documentacao probatoria RSC-III

\end{document}
\begin{center}
\textbf{\textit{{\large SUMÁRIO DO PROCESSO RSC}}}
\end{center}
%
% Tabela
%
% MUDAR 0 0(ZERO) NAS FUNCOES \DIFERENCAPAGINA PARA 1(UM) QUANDO ADICIONAR ALGUM COMPROVANTE
\begin{table}[ht]
	\centering
	\begin{tabular}{|c|c|c|}
		\hline
		\textbf{Descrição} & \textbf{Página Inicial} & \textbf{Página Final}\\
		\hline
		Planilha Simulação Pontuação & iv.1 & iv.5\\
		\hline
		Relatório Descritivo & \pageref{relatorio-descritivo}.1 & \pageref{relatorio-descritivo}.\diferencapagina{antes-documentacao-probatoria-formacao}{1}\\
		\hline
		\makecell{Documentação Comprobatória\\Formação Acadêmica e Complementar} & \pageref{documentacao-probatoria-formacao}.1 & \pageref{documentacao-probatoria-formacao}.\diferencapagina{antes-documentacao-probatoria-rsc-i}{0}\\
		\hline
		Documentação Comprobatória RSC I & \pageref{documentacao-probatoria-rsc-i}.1 & \pageref{documentacao-probatoria-rsc-i}.\diferencapagina{antes-documentacao-probatoria-rsc-ii}{0}\\
		\hline
		Documentação Comprobatória RSC II & \pageref{documentacao-probatoria-rsc-ii}.1 & \pageref{documentacao-probatoria-rsc-ii}.\diferencapagina{antes-documentacao-probatoria-rsc-iii}{0}\\
		\hline
		Documentação Comprobatória RSC III & \pageref{documentacao-probatoria-rsc-iii}.1 & \pageref{documentacao-probatoria-rsc-iii}.\diferencapagina{fim-documento}{0}\\
		\hline
	\end{tabular}
\end{table}

{\color{red}
ORIENTAÇÕES DE PREENCHIMENTO:
\begin{enumerate}
	\itemsep-1em
	\item A montagem e a organização deste processo são de exclusiva responsabilidade do docente requerente;
	\item Depois de feita a simulação de pontuação, deve-se imprimir a planilha e anexá-la nesta documentação;
	\item É necessário seguir a nomenclatura de numeração contida no sumário acima, assim cada documento deve ser numerado com um algarismo romano indicando a parte a que pertence, seguido da numeração ordinal de sua sequência; 
	\item A numeração ordinal de páginas deve ser feita na parte inferior direita;
	\item Antes das documentações comprobatórias de cada RSC deverá existir uma capa indicativa do RSC I, II e III como já consta neste documento.
	\item Cada documento comprobatório RSC deverá ter a indicação da característica de pontuação a que se se refere, por exemplo, no canto superior direito da folha escreva: RSC I – 11, assim o avaliador saberá facilmente que tal documento se refere à atuação como conferencista ou palestrante;
	\item Após anexados os documentos, não se esqueça de preencher a numeração de página final referente a cada parte deste processo, no sumário do processo RSC (tabela acima – CAMPO PÁGINA FINAL).
\end{enumerate}
}

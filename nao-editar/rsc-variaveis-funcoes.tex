\renewcommand{\cftsecdotsep}{\cftdotsep}
\renewcommand{\cftsecleader}{\normalfont\cftdotfill{\cftsecdotsep}}

\makeatletter
\newcommand\Rodape[1]{\def\@Rodape{#1}}

\fancyhead[LE,LO]{}
\fancyhead[CE,CO]{}
\fancyhead[RE,RO]{}
\fancyhead[LE,LO]{\includegraphics[width=5cm]{nao-editar/logo_ifsudeste.jpg}\hfill\includegraphics[width=2cm]{nao-editar/brasao-brasil.png}\\{\large \textbf{FORMULÁRIOS RECONHECIMENTO DE SABERES E COMPETÊNCIAS – RSC}}}
\setlength\headheight{80pt}
%\renewcommand{\headrulewidth}{0pt}
%\renewcommand{\headrulewidth}{0.4pt}
%\renewcommand{\footrulewidth}{0.4pt}

\fancyfoot{}
% numeracao a direita
\Rodape{\thepage}
\fancyfoot[R]{\@Rodape}

% para o fancy funcionar na pagina com tableofcontents
\fancypagestyle{plain}{%
	\fancyhf{}
	\fancyhead[LE,LO]{}
	\fancyhead[CE,CO]{}
	\fancyhead[RE,RO]{}
	\fancyhead[LE,LO]{\includegraphics[width=5cm]{nao-editar/logo_ifsudeste.jpg}\hfill\includegraphics[width=2cm]{nao-editar/brasao-brasil.png}\\{\large 					\textbf{FORMULÁRIOS RECONHECIMENTO DE SABERES E COMPETÊNCIAS – RSC}}}
	\setlength\headheight{80pt}
	%\renewcommand{\headrulewidth}{0pt}
	%\renewcommand{\headrulewidth}{0.4pt}
	%\renewcommand{\footrulewidth}{0.4pt}
	\fancyfoot{}
	\fancyfoot[R]{\@Rodape}
}

\newcommand\IndicacaoPontuacaoRSC[1]{\def\@IndicacaoPontuacaoRSC{#1}}

% para o fancy funcionar na pagina com tableofcontents
\fancypagestyle{plainpdf}{%
	\fancyhf{}
	\fancyhead[LE,LO]{}
	\fancyhead[CE,CO]{}
	\fancyhead[RE,RO]{}
	\fancyhead[RE,RO]{\textbf{{\Large \@IndicacaoPontuacaoRSC}}}
	%\renewcommand{\headrulewidth}{0pt}
	\fancyfoot{}
	\fancyfoot[R]{\@Rodape}
	%\setlength\headheight{0pt}
	\newgeometry{bindingoffset=0in,left=0.5in,right=0.5in,top=1in,bottom=1in,footskip=1.75in}
	%fancyhfoffset[E,O]{0pt}
}

\newcommand{\diferencapagina}[2]{%
	\number\numexpr\getpagerefnumber{#1}-#2\relax
}

\newcommand{\subtrair}[2]{%
	\number\numexpr#1-#2\relax
}

\newcommand{\nada}{
	{\color{white}NADA}
}

\newcommand{\paginainicialpdf}[1]{%
	\pageref{inicio-#1}
}

\newcommand{\paginafinalpdf}[1]{%
	\subtrair{\getpagerefnumber{fim-#1}}{1}
}

\newcommand{\incluirpdf}[4]{
	\IndicacaoPontuacaoRSC{#3}
	\label{inicio-#4}
	\includepdf[pages=#1, scale=0.93, clip, offset=0 -12, angle=#2, pagecommand={\thispagestyle{plainpdf}}]{#4}
	\label{fim-#4}
}

% \addtabelaindicacaocomprovante{RSC III}{VII}{145}{ix - 30 a ix - 33}
\newcommand{\addtabelaindicacaocomprovante}[4]{
	\begin{table}[H]
	\begin{tabular}{|c|c|c|c|}
		\hline
		\hspace{0.1in}Referência\hspace{0.1in} & \hspace{0.2in}Diretriz\hspace{0.2in} & \hspace{0.2in}Critério\hspace{0.2in} & \makecell{Página(s) do(s) documentos(s) comprobatórios (s)}\\
		\hline
		#1 & #2 & #3 & #4\\
		\hline
	\end{tabular}
\end{table}
}

% \addcaixatexto{}
\newcommand{\addcaixatexto}[1]{
	\framebox[\textwidth][c]{#1}
}

% define variaveis
\newtoks\nomeservidor
\newtoks\datanascimento
\newtoks\siape
\newtoks\email
\newtoks\nivel
\newtoks\classe
\newtoks\dataingressoservicopublico
\newtoks\dataingressoif
\newtoks\formacao
\newtoks\tempoefetivoexercicio
\newtoks\rscpretendida
\newtoks\cpf
\newtoks\campus
\newtoks\instituicao

% define as variaveis pontuacoes
% RSC-I
\newtoks\rscipontosi
\newtoks\rscipontosii
\newtoks\rscipontosiii
\newtoks\rscipontosiv
\newtoks\rscipontosv
\newtoks\rscipontosvi
\newtoks\rscipontosvii
\newtoks\rscipontosviii
%\newtoks\rscipontostotal

% RSC-II
\newtoks\rsciipontosi
\newtoks\rsciipontosii
\newtoks\rsciipontosiii
\newtoks\rsciipontosiv
\newtoks\rsciipontosv
\newtoks\rsciipontosvi
\newtoks\rsciipontosvii
\newtoks\rsciipontostotal

% RSC-II
\newtoks\rsciiipontosi
\newtoks\rsciiipontosii
\newtoks\rsciiipontosiii
\newtoks\rsciiipontosiv
\newtoks\rsciiipontosv
\newtoks\rsciiipontosvi
\newtoks\rsciiipontosvii
\newtoks\rsciiipontostotal

